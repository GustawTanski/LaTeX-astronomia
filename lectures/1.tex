% !TeX root = ../index.tex
\documentclass[../index.tex]{subfiles}
\begin{document}
    \section{Organizacja}
        \begin{itemize}
            \item Egzamin ustny
            \item https://openstax.org/details/books/astronomy \--- przydatna i darmowa książka
            \item Marcin Kubika \--- ,,Gwiazdy i materia międzygwiazdowa''.
        \end{itemize}
    \section{Wykład 1}
        Doba nie jest równa obrotowi Ziemi o \(360\degree\), a o ok. \(361\degree\) ponieważ składają się ze sobą dwa ruchy \textbf{obiegowy} i \textbf{obrotowy} ziemi. Na niebie można dostrzec księżyc, słońce, gwiazdy i planety. Te ostatnie od gwiazd odróżniają się tym, że przemieszczają się po niebie. Obserwacje astronomiczne dały ludziom możliwość precyzyjnego pomiaru czasu. Dalej obserwując gwiazdy można orientować się w terenie (gwiazda polarna, krzyż południa).\\
        \textbf{Gwiazdozbiory} to umównie połączone ze sobą w grupy gwiazdy, w celu łatwego zidentyfikowania ich na niebie. Dokładniej jest to pewien obszar o ściśle określonych granicach. Nie wszystkie gwiazdozbiory widać z jednego punktu na Ziemi. Są takie, które widać zawsze. Inne widać okresowo (w niektórych porach roku). Jeszcze innych nigdy nie widać (widać je tylko z drugiej półkuli). Obecnie wyróżnia się 88 gwiazdozbiorów. \\
        \textbf{Nazwy gwiazd} \--- niektóre gwiazdy mają swoje nazwy gwiazdy. Inną metodą jest oznaczanie gwiazd w gwiazdozbiorze kolejnym literami greckim lub liczbami (ze względu na położenie lub jasność).
        Obserwowane gwiazdy maję różne wielkości. Skala wielkości gwiazd nazywane jest \textbf{magnitudo}. Nie należy wiązać obserwowaną wielkość gwiazd z ich rzeczywistymi rozmiarami, a ich jasnością. Rzeczywiście gwiazdy są punktami. Związek pomiędzy magnitudo a jasnością wyraża \textbf{wzór Pogsona}:
        \begin{equation}
            m_A - m_B = - 2.5 \log \frac{F_A}{F_B} 
        \end{equation} 
        Do stosowania powyższego wzoru wymaga zdefiniowania pewnego standardu, względem którego wyznacza się jasność innych gwiazd. Obecnym standardem jest \textbf{Wega} (\(\alpha\)\! Lyr). Oprócz tego istnieje więcej rozsianych po niebie standardów, aby łatwo określać wielkość gwiazdową na dowolnym punkcie nieba. Różnica 5 magnitudo to różnica dwóch rzędów wielkości jasności gwiazdy.\\
        Bardzo długo nie umiano określać odległości do gwiazd. Co prawda obserwacja paralaksy umożliwiałaby oszacowanie odległości, ale do tego potrzebne były dobre teleskopy oraz technika fotografii, ponieważ jest to efekt subtelny. Zasięg obserwowania paralaksy z ziemi to 100 parseków. Jeden \textbf{parsek} jest to odległość z jakiej jednostka astronomiczna jest obserwowana jako jedna sekunda łuku. Natomiast z orbity 1000 pc (Hipparcos) aż do 10000 pc (GAIA).
\end{document}
