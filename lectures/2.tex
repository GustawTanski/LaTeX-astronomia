% !TeX root = ../index.tex
\documentclass[../index.tex]{subfiles}

\begin{document}
    \section{Wykład 2}
        \subsection{Paralaksa}
            Obserwują jeden obiekt (np. gwiazdę) z dwóch różnych perspektyw można zaobserwować zmianę jego położenia względem bardziej odległego tła. Jeśli nakreślimy trójkąt pomiędzy początkowym położeniem, końcowym położeniem oraz obserwowanym obiektem i oznaczymy \(\pi^{0}\) kąt przy obserwowanym obiekcie to odległość od tego punktu do przeciwległego boku jest określona wzorem:
            \begin{equation}
                D = b \frac{57.3\degree}{\pi^{0}} \ref{eq:paralax_1}
            \end{equation}
            gdzie \(b\) to długość łuku pomiędzy dwoma punktami obserwacji (w przypadku pomiarów astronomicznych z ziemi, to ten łuk jest w przybliżeniu prostą łączącą dwa położenia ziemi).\\
            W pomiarach astronomicznych kąt \(\pi^{ 0}\) jest zwykle bardzo mały. Stopień jest w tym przypadku zbyt dużą jednostką, więc stosuje się miarę sekund łuku. Ponadto do określania \(b\) używa się jednostek astronomicznych. Wówczas wzór \label{eq:paralax_1}
            \begin{equation}
                D[AU] = \frac{206.65''}{\pi^{0}}  
            \end{equation}
            Jeśli natomiast wynik miałby być otrzymany w \textbf{parsekach} to powyższy wzór przybiera postać:
            \begin{equation}
                D[pc] = \frac{1}{\pi^{0}}
            \end{equation}
            Bierze się to z definicji parseka \--- odległość dla której kąt paralaksy jest równy jednej sekundzie łuku.\\
            Pojedyncze pomiary paralaksy mają mały sens ze względu na dużą niepewność pomiarów. Dodatkowym utrudnieniem jest ruch własny obserwowanych gwiazd względem układu słonecznego. Jedynie duże ilość tych pomiarów ma sens. Pierwsze pomiary paralaksy to rok 1838, ponieważ dopiero wtedy sprzęt obserwacyjny był wystarczająco dobry.
        \subsection{}
            Znając odległość gwiazdowe można było szacować ich jasności. W rezultacie odkryto, że w sąsiedztwie naszego układu gwiezdnego znajdują się głównie gwiazdy słabo świecące, zatem można się spodziewać, że wszędzie indziej sytuacja może być podobna.
        \subsection{Jasność absolutna}
            Oprócz jasności obserwowalnej (\(m\)) definiuje się też jasność absolutną (\(M\)) (opisywaną także w skali magnitudo). Jest to jasność obserwowana gwiazdy w odległości \(10 pc\). Różnica między jasnościami dana jest wzorem:
            \begin{equation}
                m - M = 5 \log D[pc] - 5 \label{eq:absolute_magnitudo}
            \end{equation}
        \subsection{Nośniki informacji}
            Nośnikami informacji z kosmosu mogą być: \textbf{fale elektromagnetyczne}, \textbf{cząstki}, \textbf{neutrina} oraz \textbf{fale grawitacyjne}. Cząstki oddziałują z polami sił (grawitacja i elektromagnetyzm), więc mogą być silnie odginane. Szczególnie biorąc pod uwagę fakt istnienia wewnątrz galaktyk a nawet międzygalaktycznych pól elektromagnetycznych. Neutrina bardzo słabo oddziałują co przekłąda się na to, że są bardzo przenikliwe. Np. słońce jest silnie nieprzepuszczalne dla światła, ale już dla neutrin nie. W związku z tym mierząc neutrina można uzyskiwać informacje na temat wydarzeń wewnątrz jąder gwiazd. Natomiast nie wszędzie można z nich korzystać. Fale grawitacyjne zostały odkryte najpóźniej i ze względu trudność pomiaru używane są do pomiarów zjawisk na ogromną skalę \--- takich jak łączenia czarnych dziur lub gwiazd neutronowych.
            \subsubsection{Fale elektromagnetyczne}
                W astronomii korzysta się z fal elektromagnetycznych z niemal całego spektrum. Fotony te różnią się nawet o kilkanaście rzędów wielkości jeśli chodzi o energie. Najpierw obserwowane były jedynie fale w spektrum widzialnym. Następnie technika fotografii poszerzyła to spektrum (głównie podczerwone). Dalej człowiek nauczył się rejestrować fale radiowe. Ze względu na to, że pozostałe zakresy są odbijane lub pochłaniane przez atmosferę ziemi, zaczęły być one rejestrowane dopiero w epoce kosmicznej. Część promieniowania pochłaniane jest przez parę wodną zawartą w atmosferze \--- dlatego obserwatoria astronomiczne najlepiej lokować w suchych obszarach.\\
                Możliwa jest też pośrednia obserwacja wysokoenergetycznych kwantów \(\gamma\). Taki foton wpadając w atmosferę i reagując z materią tworzy efekt kaskadowy (\textbf{pęk atmosferyczny}). Część cząstek obdarzonych energią przez taką cząstkę może poruszać z prędkością większą niż prędkość światła w atmosferze. W związku z tym następuje efekt Czerenkowa w zakresie niskiego nadfioletu. I takie światło można już obserwować.
\end{document}