% !TeX root = ../index.tex
\documentclass[../index.tex]{subfiles}

\begin{document}
    \section{Wykład}
        \subsection{Galaktyki}
            Wiele obiektów, które początkowo uważano za mgławice, okazało się być innymi niż droga mleczna galaktykami. Podstawowa klasyfikacja galaktyki (\textit{widelec Hubble'a}) dzieli je na \textbf{eliptyczne}, \textbf{spiralne}, \textbf{spiralne z poprzeczką}, \textbf{nieregularne} i \textbf{osobliwe} (\textbf{aktywne}).\\
            Galaktyki aktywne to młode galaktyki, w których emisja zdominowana jest przez procesy zachodzą w pobliżu horyzontu zdarzeń supermasywnej czarnej dziury w centrum galaktyki, w przeciwieństwie do reszty galaktyk, w których świecą głównie gwiazdy. Prostopadle do osi obrotu czarnej dziury znajduje się torus zbudowany z pyłów, a w wąskim stożku wokół osi rotacji emitowana jest wąska strugo relatywistycznej materii zwana \textbf{dżetem}.\\
            Około połowa galaktyk to galaktyki nieregularne, które są wynikiem zderzania się galaktyk. Prawdopodobnie część galaktyk eliptycznych ma podobny rodowód.\\
            Same zderzenia można licznie obserwować. W obserwowalnym wszechświecie liczba galaktyk jest na poziomie kilkuset miliardów.
        \subsection{Grupowanie się galaktyk}
            Galaktyki grupują się w różnego rodzaju struktury. Wyróżniamy kolejno: \textbf{grupy}, \textbf{gromady} i \textbf{supergromady} galaktyk. W jeszcze większej skali można wyróżnić \textbf{ściany} lub inaczej \textbf{włókna}, pomiędzy którymi znajdują się, znacznie mniej gęsto wypełnione materią, \textbf{pustki}. Grupy zawierają nie więcej niż 50 galaktyk i nie stanowią trwałego związku grawitacyjnego. Nasza galaktyka należy do \textbf{Lokalnej Grupy Galaktyk}, w której wraz z galaktyką \textbf{M31} (Andromeda) jest największym składnikiem. Gromady zawierają do kilku tysięcy galaktyk i tworzą stabilny układ grawitacyjny. Supergromady składają się z setek tysięcy galaktyk, a ich rozmiary przekraczają nawet \(100\:Mpc\). Supergromada, do której należymy nosi nazwę \textbf{Lokalna} lub \textbf{Virgo} (bo centralne gromada jest zlokalizowana w gwiazdozbiorze Panny).\\
            Masowe przeglądy fotometryczne nieba doprowadziło do odkrycia jeszcze większych struktur – włókien (ścian), które otaczają olbrzymie bąble zwane pustkami, w których gęstość występowania galaktyk jest znacznie niższa. Uważa się, że są to największe struktury, którymi jednolicie wypełniony jest cały Wszechświat i że jest to pozostałość po \textbf{barionowych oscylacjach akustycznych}, które zachodziły w bardzo wczesnym stadium ewolucji Wszechświata.
        \subsection{Prawo Hubble'a i Wielki Wybuch}
            Edwin Hubble zauważył, że im dalej od obserwator znajduje się badana galaktyka, tym szybciej się od niego oddala, niezależnie w którym kierunku patrzeć. Sformułował \textbf{prawo Hubble'a}:
            \begin{equation}
                v = H_0 D
            \end{equation}
            gdzie \(H_0\) – stała Hubble'a, \(D\) – odległość od galaktyki. Skoro galaktyki stale oddalają się od siebie, to kiedyś w przeszłość musiały być wszystkie w jednym punkcie. Zdarzenie rozpoczynające proces oddalania się galaktyk nosi miano \textbf{Wielkiego Wybuchu}. Dokładne wyliczenia stałej Hubble pozwalają na oszacowanie wieku wszechświata 13.8 mld lat.
        \subsection{Mikrofalowe promieniowanie tła i początki Wszechświata}
            Wszechświat jest wypełniony tzw. \textbf{mikrofalowym promieniowaniem tła}, które jest promieniowanie o rozkładzie Planckowskim i temperaturze \(2.725\: K\). Nie ważne w którym kierunku skierujemy radioteleskop wykrywamy je. Powstało około 380 tysięcy lat po wielkim wybuchu, kiedy temperatura Wszechświata spadła poniżej \(3000\:K\) co umożliwiło łączenie się elektronów z protonami w atomach i skutkowało oddzieleniem promieniowania od materii. Istniejące wówczas fotony rozpierzchły się po wszechświecie (średnia droga swobodna porównywalna z rozmiarami Wszechświata) stopniowo stygnąć do obecnej temperatury. Epoka, która nastąpiła po tym wydarzeniu nosi miano \textbf{wieków ciemnych}, ponieważ dopiero po kilkaset milionach lat zaczęły powstać pierwsze gwiazdy, więc we Wszechświecie nie było żadnych źródeł fotonów.\\
            W promieniowaniu tła można dostrzec fluktuacje, które zgadzają się z fluktuacjami gęstości materii we wszechświecie.\\
            Szczegółowe badanie rozkładu promieniowania tła daje możliwość oglądania Wszechświata 380 tys. lat po Wielkim Wybuchu i stanowi dziś jedno z najważniejszych narzędzi testowania modeli kosmologicznych. Dostarcza również wiedzy o wieku i strukturze Wszechświata.
        \subsection{Zmienne tempo ekspansji wszechświata}
            Obserwacja supernowych typu Ia w odległych galaktykach wskazuje, że tempo ekspansji wszechświata się zwiększa. Jednakże używanie \textbf{błysków promieniowania gamma} jako świec standardowych w jeszcze bardziej odległych galaktykach wskazuje, że w pewnym okresie ekspansja zwalniała. Trend ze zwalniającego na przyspieszający zmienił się około 5 mld. lat temu.
            \subsubsection{Ciemna energia}
                Jako przyczynę rosnącego tempa ekspansji Wszechświata uważa się \textbf{ciemną energię}. Jest to hipotetyczna forma energii, która miałaby wypełniać całą przestrzeń i wywierać ujemne ciśnienie, przez co wywoływać efekt odpychania się galaktyk. Statystyczna interpretacja ciemnej energii to \textbf{stała kosmologiczna}. Przy założeniu jej niezmienności, zgodność z obserwacjami można uzyskać, przyjmując, że średnia gęstość materii we Wszechświecie spadła na skutek jego ekspansji poniżej gęstości ciemnej energii właśnie 5 mld. lat temu.\\
                Pole kwantowe związane z ciemną energią nosi nazwę \textbf{kwintesencji}.
            \subsection{Inflacja}
                Bardzo krótko po Wielkim Wybuchu nastąpiła \textbf{inflacja} – gwałtowne i krótkie rozszerzanie się wszechświata (znacznie szybsze niż prędkość światła).
            \subsection{Przyszłość Wszechświata}
                Obecnie brane są trzy możliwe scenariusze tego co stanie się z Wszechświatem w przyszłości: \textbf{Big Rip} – tempo ekspansji będzie nadal rosnąć co doprowadzi do rozerwania Wszechświata, \textbf{Big Freeze} – tempo ekspansji będzie rosnąć wolniej niż w przypadku Big Rip i skończy sie schłodzeniem całego wszechświata do temperatury \(0\:K\) i \textbf{Big Crunch} – ekspansja zacznie zwalniać, aż się odwrócić ostatecznie dojdzie do kolapsu z powrotem do początkowej osobliwości.
        \subsection{Badania Wszechświata}
            Na podstawie analizy omówionych wyżej materiałów obserwacyjnych (hierarchiczna struktura rozmieszczenia materii, fluktuacje mikrofalowego promieniowania tła, rosnące tempo ekspansji Wszechświata) możliwe jest określenie geometrii Wszechświata oraz globalnych proporcji pomiędzy ciemną energią, ciemną materią i zwykłą materią.
\end{document}